% !TeX root = document.tex
% !TeX program = lualatex

% !TeX root = document.tex
\documentclass[12pt]{report}

\input{build/cmake}

\usepackage{polyglossia}
\setmainfont{CMU Serif}
\setsansfont{CMU Sans Serif}
\setmonofont{JuliaMono-Regular.ttf}[
	Path = resources/fonts/JuliaMono/,
	BoldFont = JuliaMono-Bold.ttf,
	ItalicFont = JuliaMono-LightItalic.ttf,
	BoldItalicFont = JuliaMono-BoldItalic.ttf ]
\setmainlanguage{ukrainian}
\setotherlanguages{english, russian}

\usepackage{geometry}
\geometry{
	paper=a4paper,
	inner=10mm, outer=10mm, top=20mm, bottom=20mm, 
	bindingoffset=0mm,
	headheight=6mm,
	includehead=false, includefoot=false, portrait,
	% twoside
}

\usepackage{graphicx}
\graphicspath{{images/}}
\usepackage{subcaption}
\usepackage{float}
\usepackage{wrapfig}

\usepackage[svgnames]{xcolor}
\usepackage{blindtext}
\usepackage[useregional]{datetime2}
\usepackage{syntonly}

\usepackage{array}
\renewcommand{\arraystretch}{1.5}
\usepackage{booktabs}
\usepackage{nicematrix}
\usepackage{multirow}
\usepackage[originalparameters]{ragged2e}
\usepackage{changepage}
\usepackage{verbatim}

\usepackage{amsmath}
\allowdisplaybreaks[3]
\usepackage{cancel}
\usepackage{mathtools}
\usepackage{amsfonts}
\usepackage[e]{esvect}
\numberwithin{equation}{chapter}
\DeclareMathOperator{\rank}{rank}
\DeclareMathOperator{\lcm}{lcm}
\DeclareMathOperator{\pr}{pr}
\DeclareMathOperator{\arccot}{arccot}
\DeclareMathOperator{\grad}{grad}
\DeclareMathOperator{\med}{MED}
\DeclareMathOperator{\mad}{MAD}
\DeclareMathOperator{\cov}{cov}

\usepackage{enumitem}
\setlist{noitemsep, itemsep=1pt, topsep=1pt, labelsep=2pt}

\usepackage{caption}
\captionsetup{%
	skip=-2mm, 
}

\usepackage{titlesec}
% \titleformat{\chapter}[display] % default chpater definition
% {\normalfont\huge\bfseries}{\chaptertitlename\ \thechapter}{20pt}{\Huge}
% \titlespacing*{\chapter}{0pt}{0pt}{40pt} % 0pt 50pt 40pt
\usepackage{extramarks}
\usepackage{fancyhdr}
\fancyhf{}
\fancyhead[R]{\nouppercase{\firstleftmark}} 
\fancyfoot[C]{\ttfamily\thepage{}}
\fancyfoot[R]{\ttfamily\shortfaculty{}}
\fancyfoot[L]{\ttfamily\shortdepartment{}~\shortuniversity{}}
\pagestyle{fancy}
\fancypagestyle{plain}{%
	\fancyhf{}%
	\fancyfoot[C]{\ttfamily\thepage{}}%
	\fancyfoot[R]{\ttfamily\shortfaculty{}}%
	\fancyfoot[L]{\ttfamily\shortdepartment{}~\shortuniversity{}}%
	\renewcommand{\headrulewidth}{0pt}%
}

\usepackage[outputdir=build/aux]{minted}
\usemintedstyle{gruvbox-light}
\setminted{%
	linenos=true,
	autogobble,
	breaklines,
	tabsize=2,
	bgcolor=GhostWhite,
	fontsize=\scriptsize,
	curlyquotes=true,
	mathescape,
}
\setmintedinline{%
	bgcolor=White,
	mathescape,
}
\newmintinline[cinl]{c}{}
\newmintinline[cppinl]{cpp}{}
\newmintinline[pyinl]{py}{}
\newmintinline[csinl]{cs}{}
\newmintinline[luainl]{lua}{}
\newmintinline[sqlinl]{sql}{}
\newmintinline[textinl]{text}{}
\newenvironment{longlisting}{\captionsetup{type=listing}}{}

\usepackage[autostyle]{csquotes}
\usepackage[nottoc]{tocbibind}
\usepackage[sorting=none]{biblatex}

\usepackage[titletoc, toc, title]{appendix}

\usepackage{imakeidx}
\indexsetup{firstpagestyle=fancy}
\makeindex[options=-s iso]

\usepackage{hyperref}
\usepackage{cleveref}
\hypersetup{%
	colorlinks=true,
	linkcolor=PaleVioletRed,
	citecolor=PaleVioletRed,
	raiselinks=true,
	pdfview=XYZ,
	pdfviewarea=TrimBox,
	linktocpage=true,
	urlcolor=Teal,
}

\usepackage{xr}
\usepackage{subfiles}
\externaldocument[M-]{\subfix{document}}

\counterwithin{listing}{chapter}
% \setcounter{tocdepth}{5}
% \setcounter{secnumdepth}{3}

\addto\captionsukrainian{%
  \renewcommand{\listingscaption}{Лис.}%
  \renewcommand{\listoflistingscaption}{Перелік листингів}%
	\renewcommand{\appendixtocname}{Додатки} 
	\renewcommand{\appendixpagename}{Додатки}
}


\newcommand{\crule}{\par\noindent\rule{\textwidth}{0.5pt}}

\begin{document}
\thispagestyle{empty}

\begin{center}
	\LARGE \textbf{Владислав Реган}\\[4mm] \normalsize
	\texttt{\phonenumber{+380636637248} \(\diamond\)\
	\href{mailto:rehanvladyslav@gmail.com}{rehanvladyslav@gmail.com} \(\diamond\)
	\url{https://github.com/Dolfost}}
\end{center}
\crule
\section{Освіта}
\textbf{Прикладна математика | ОП прикладне ПЗ} | Бакалавр, третій курс | \href{http://nau.edu.ua/en/}{Київський Авіаційний Інститут}\\
Поступив у університет на платну форму навчання, перевівся на державну після першого курсу.
Середня оцінка за попередній семестр рівна 87.37 із 100.
Курсові роботи включають:
\begin{itemize}
	\item Статичтичний підхід до виявлення малоімовірних фрагментів цифрового зображення
	\item Програмне забезпечення для аналізу багатовимірних данних
\end{itemize}

\crule
\section{Про себе}
Можу вевнено читати літературу та документацію англійською мовою. Працював з
Github Actions, маю базовий досвід використання FreeCAD. В основному пишу на
C++. Більшість знань про програмування та зв'язаних з ним інструментів було
освоєно із
\href{https://github.com/Dolfost/Dolfost/blob/main/README.md#literature--sources}{книжок}
та документацій. Користуюся LaTeX (lualatex) для звітів по університету на
постійній основі.

\crule
\section{Досвід роботи}
Досвіду роботи не маю.

\section{Навички}
\begin{itemize}
	\item C/C++, CMake, Bash, Lua, LaTeX
	\item Git, Doxygen
\end{itemize}
C++ бібліотеки: STL, Qt6, ExprTk, QCustomPlot, libpng, tomlplusplus

\section{Інструменти}
Користуюся \href{https://github.com//Dolfost/dotfiles}{конфігурованим} tmux,
nvim, lldb, засобами командного рядка *nix. Маю тижневий досвід користування
QtCreator який я отримав перед тим як повністю перейти на CMake базований
робочий процесс. Розумію концепти модульного тестування CTest та семантики
пакування програмного забезпечення за допомогою CPack. Регулярно користуюся
технологіями віддаленого терміналу по типу ssh, scp(rsync). Володію локальною
лінукс машиною яка доступна із глобальної мережі.

\section{Особисті проекти}
\begin{itemize}
	\item \href{https://github.com/Dolfost/matstat}{Програмне забезпечення для
		аналізу багатовимірних даних} (C++, Qt, QCustomPlot, ExprTk, \(\approx
		12\)k ліній)
	\item \href{https://github.com/Dolfost/tartan}{Бекенд гри у шахмати} (C++,
		CI/CD, Doxygen, CTest, CPack)
	\item \href{https://github.com/Dolfost/ssmk/tree/develop}{.png фабрика спрайт
		листів} (C++, libpng, tomlplusplus)
	\item \href{https://github.com/Dolfost/calgo}{Маленька бібліотека по чисельих
		методах} (C++, CI/CD, Qt, Doxygen) 
	\item \href{https://github.com/Dolfost/snake-cpp}{Грау у змійку із текстовим
		інтерфейсом} (C/C++, newcurses текстовий інтерфейс користувача)
\end{itemize}

\vfill{}

\begin{center}
	\tiny \href{\lrthomepage}{git ver. \lrtversion}
\end{center}

\end{document}
