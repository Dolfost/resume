\documentclass[../../document]{subfiles}

% Постановка задачі

\begin{document}
\graphicspath{{images/}}

\chapter{Постановка задачі}
\textbf{Тема:} \worktheme.\\
\textbf{Мета:} \workobjective.

\paragraph*{Загальні вимоги до програми}
\begin{enumerate}
	\item Програма повинна бути незалежна від даних. Вхідний файл має
		обиратися в діалозі з користувачем. Передбачається, що вхідні дані знаходяться
		в текстовому файлі, обсяг даних не відомий. Потрібно забезпечити можливість
		модифікації та збереження даних.
	\item Слід уможливити перетворення даних (логарифмування,
		стандартизація, зсув).
	\item Після перетворення або вилучення аномальних значень користувач
		повинен мати можливість повернутися до початкових даних.
\end{enumerate}

\blindtext\footnote{This is a footnote}

\begin{figure}
	\begin{center}
		\fbox{\includegraphics[width=0.5\textwidth]{../../resources/images/pm.png}}
	\end{center}
	\caption{Faculty image}\label{fig:pm}
\end{figure}

\blindtext

\begin{longlisting}
	\begin{minted}{c++}
		// Header file for input output functions
		#include <iostream>
		using namespace std;

		// main() function: where the execution of
		// C++ program begins
		int main() {

				// This statement prints "Hello World"
				cout << "Hello World";

				return 0;
		}
	\end{minted}
	\caption{C++ hellow world}\label{lst:code}
\end{longlisting}
This is an \cppinl{int main()} function at business. \blindtext

\begin{table}
	\begin{center}
		\begin{NiceTabular}{cccc}
			\CodeBefore
			\rowlistcolors{1}{Lavender,}
			\Body
			\toprule
			this & is & a & table \\
			\midrule
			and & this is & it's & middle\\
			and & this is & it's & middle\\
			and & this is & it's & middle\\
			and & this is & it's & middle\\
			and & this is & it's & middle\\
			and & this is & it's & middle\\
			and & this is & it's & middle\\
			\bottomrule
		\end{NiceTabular}
	\end{center}
	\caption{Table}
\end{table}

\end{document}
